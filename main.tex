\documentclass{beamer}

\usepackage{fancyhdr}
\usepackage[quiet]{fontspec}
\usepackage{tikz}
\usepackage{almslides}
\usepackage{moresize}

\usetikzlibrary{chains,decorations.pathmorphing,positioning,fit}
\usetikzlibrary{decorations.shapes,calc,backgrounds}
\usetikzlibrary{decorations.text,matrix}
\usetikzlibrary{arrows,shapes.geometric,shapes.symbols,scopes}
\usetikzlibrary{mindmap}

\setmonofont{Monaco}

\title{A Glance at MongoDB}
\author{Sheng Yuan}

\begin{document}
    
\begin{frame}
	\centering\includegraphics[width=0.7\textwidth]{mongodb-logo.png}
    \titlepage
\end{frame}

\begin{frame}{Contents}
    \tableofcontents
\end{frame}

\section{Introduction}
\begin{frame}{Introduction}
	\begin{block}{Document Database}
		A record in MongoDB is a document, which is a data structure composed of field and value pairs. MongoDB documents are similar to JSON objects. The values of fields may include other documents, arrays, and arrays of documents.
	\end{block}
	\begin{block}{Key Features}
		\begin{itemize}
			\item High Performance
			\item Rich Query Language
			\item High Availability
			\item Horizontal Scalability
			\item Support for Multiple Storage Engines
		\end{itemize}
		
	\end{block}
    
\end{frame}

\section{Data Model}
\begin{frame}{Concepts}

\begin{minipage}[t]{0.4\textwidth}
	\begin{Alms*}
		> Documents \\
		> Collections \\
		> Database \\
		
%		\A{| Mongodb} \A{| MySql} \\
%		\A{| Document} \A{| Row} \\
%		\A{| Field} \A{| Column} \\
%		\A{| Collection} \A{| Table}\\
%		\A{| Database} \A{| Database}\\
	\end{Alms*}
\end{minipage}
\hfill
\begin{minipage}[t]{0.6\textwidth}
	\scriptsize
	\begin{Alms*}
		> \K{show} \V{dbs} \\
		admin \qquad 0.000GB\\
		local \qquad 0.000GB\\
		firebase \,0.001GB\\
		> \K{use} \V{firebase} \\
		switched to db firebase \\
		> \K{show} \V{collections} \\
		users \\
		models \\
		projects \\
		tasks \\
		> \V{db}.\V{users}.\V{find}().\V{pretty}() \\
		\{ \NI
		"\_id":ObjectId, \\
		"email":"liueh@tcl.com", \\
		"role":"admin"
		\ND \} \\
		\{ \NI
		"\_id":ObjectId, \\
		"email":"yuansheng@tcl.com", \\
		"role":"user"
		\ND \}
	\end{Alms*}
\end{minipage}

\end{frame}

\begin{frame}{Concepts}
	
\end{frame}

\begin{frame}{Data Types}
    \begin{itemize}
        \item Null
        \item Boolean
        \item Integer
        \item Double
        \item String
        \item Date
        \item Array
        \item Embedded Document
        \item ObjectId
        \item BinData
    \end{itemize}
\end{frame}

\section{Working with Data}
\begin{frame}{Mongo Shell}
    The mongo shell is an interactive JavaScript interface to MongoDB. 
    \begin{minipage}[t]{0.5\textwidth}
    	\scriptsize
    	\begin{Alms*}
    		\$ mongo \\
    		MongoDB shell version v3.4.3 \\
    		> \K{show} \V{dbs} \\
    		admin \qquad 0.000GB\\
    		local \qquad 0.000GB\\
    		firebase \,0.001GB\\
    		> \K{use} \V{firebase} \\
    		switched to db firebase \\
    		> \K{show} \V{collections} \\
    		users \\
    		models \\
    		projects \\
    		tasks \\
    		> \V{db}.\V{users}.\V{find}(\{\\
    		"email":"liueh@tcl.com"\\
    		\}).\V{pretty}() \\
    		\{ \NI
    		"\_id":ObjectId, \\
    		"email":"liueh@tcl.com", \\
    		"role":"admin"
    		\ND \}
    	\end{Alms*}
    \end{minipage}
\end{frame}

\begin{frame}{Insert}

\end{frame}

\begin{frame}{Delete}
    
\end{frame}

\begin{frame}{Update}
    
\end{frame}

\begin{frame}{Query}
    
\end{frame}

\section{Advanced Queries}
\begin{frame}
    
\end{frame}

\section{Python and MongoDB}
\begin{frame}{Python and MongoDB}
\begin{Alms*}
	\T{int} \V{udp\_sendmsg}(\T{struct sock *}\V{sk},
	\T{struct msghdr *}\V{msg}, \textrm{\ldots}) \\
	\{ \NI
	\vdots \\
	\tikzanchor{lock 1}%
	\only<8->{\highlight<8-9>{\V{lock\_sock}(\V{sk});} \\}%
	\K{if} \highlight<6-8>{(\V{unlikely}(\V{sk}$→$\V{pending}))} \{ \NI
	\highlight<5>{
		\CCOM{Socket is already corked while preparing it} \\
		\CCOM{\ldots\,which is an evident application bug. --ANK}
	} \\
	\only<9->{\highlight<9>{\V{release\_sock}(\V{sk});} \\}%
	\V{LIMIT\_NETDEBUG}(\V{KERN\_DEBUG} \S{udp cork app bug 2}); \\
	\highlight<6>{\K{return} -\D{EINVAL};}
	\ND\} \\[4pt]
	\tikzanchor{lock 2}%
	\only<-7>{\highlight<2,7>{\V{lock\_sock}(\V{sk});} \\}%
	\highlight<3>{%
		\V{ret} = \V{ip\_append\_data}(\V{sk}, \V{msg}$→$\V{msg\_iov},
		\V{ulen}, \textrm{\ldots});} \\
	\vdots \\
	\highlight<4>{\V{release\_sock}(\V{sk});} \\
	\K{return} \V{ret};
	\ND\}
\end{Alms*}
\end{frame}

\section{Golang and MongoDB}
\begin{frame}{Golang and MongoDB}

\vspace{0.04\textheight}
\end{frame}

\section{GridFS}
\begin{frame}{GridFS}
	GridFS is a specification for storing and retrieving files that exceed the BSON-document size limit of 16 MB.
	\vspace{0.05\textheight}
	
	\begin{minipage}[t]{0.45\textwidth}
		%\includegraphics[width=\linewidth]{golang}
		Instead of storing a file in a single document, GridFS divides the file into parts, or chunks, and stores each chunk as a separate document. By default, GridFS uses a chunk size of 255 kB; that is, GridFS divides a file into chunks of 255 kB with the exception of the last chunk.
	\end{minipage}%
	\hfill
	\begin{minipage}[t]{0.45\textwidth}
		\scriptsize
		\begin{Alms*}
		\{ \NI
			"email":"yuansheng@tcl.com", \\
			"role":"admin"
		\ND \}
		\end{Alms*}
	\end{minipage}
\end{frame}

\begin{frame}{GridFS}
	\begin{minipage}[t]{\textwidth}
		%\includegraphics[width=\linewidth]{golang}
		GridFS uses two collections to store files. One collection stores the file chunks, and the other stores file metadata.
	\end{minipage}
\end{frame}

\section{Database Administration}
\begin{frame}
    
\end{frame}

\section{Optimization}
\begin{frame}
    
\end{frame}

\section{MonogDB and MySQL}
\begin{frame}
    
\end{frame}

\end{document}
